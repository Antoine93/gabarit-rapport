% ============================================================
% DÉVELOPPEMENT
% ============================================================
% Instructions pour rédiger cette section :
% - Développez le contenu principal de votre rapport
% - Divisez en sous-sections logiques
% - Incluez formules, tableaux, figures et code selon vos besoins
% - Référencez les exemples dans sections/examples/ pour la syntaxe
% ============================================================

\section{Développement}
\label{sec:developpement}

% TODO: Remplacez ce contenu par votre développement

Cette section présente le développement du sujet.

\subsection{Première sous-section}

% TODO: Développez votre première partie

Vous pouvez inclure des formules mathématiques :
\begin{equation}
	E = mc^2
	\label{eq:exemple}
\end{equation}

Ou des formules inline comme $F = ma$.

\subsection{Deuxième sous-section}

% TODO: Développez votre deuxième partie

\subsubsection{Méthodologie}

% TODO: Expliquez votre méthodologie

\subsubsection{Mise en œuvre}

% TODO: Décrivez la mise en œuvre

% Exemple de tableau (référez-vous à sections/examples/tableaux.tex)
\begin{table}[H]
	\centering
	\caption{Exemple de tableau}
	\label{tab:exemple}
	\begin{tabular}{|l|c|c|}
		\hline
		\textbf{Paramètre} & \textbf{Valeur} & \textbf{Unité} \\
		\hline
		Paramètre 1 & 10 & m \\
		Paramètre 2 & 20 & kg \\
		\hline
	\end{tabular}
\end{table}

% Exemple de figure (référez-vous à sections/examples/figures.tex)
\begin{figure}[H]
  \centering
  \includegraphics[width=0.6\textwidth]{images/votre_image.png}
  \caption{Description de la figure}
  \label{fig:exemple}
\end{figure}

% Exemple de code MATLAB (référez-vous à sections/examples/code.tex)
\begin{lstlisting}[caption={Exemple de code}, label={code:exemple}]
% Votre code MATLAB ici
x = 1:10;
y = x.^2;
plot(x, y);
\end{lstlisting}

\subsection{Troisième sous-section}

% TODO: Développez votre troisième partie
