\documentclass[6pt,landscape]{extarticle}
\usepackage[utf8]{inputenc}
\usepackage[T1]{fontenc}
\usepackage[top=0.5cm, bottom=0.5cm, left=0.5cm, right=0.5cm]{geometry}
\usepackage{multicol}
\usepackage{enumitem}
\usepackage{tabularx}
\usepackage{array}
\usepackage{booktabs}
\usepackage{xcolor}
\usepackage{titlesec}
\usepackage{amsmath,amssymb,amsfonts}
% Configuration compacte
\setlength{\parindent}{0pt}
\setlength{\parskip}{0pt}
\setlength{\columnsep}{0.2cm}
\pagestyle{empty}
% Titres compacts
\titlespacing{\section}{0pt}{1pt}{0.5pt}
\titlespacing{\subsection}{0pt}{0.5pt}{0pt}
\titleformat{\section}{\normalfont\normalsize\bfseries}{\thesection}{0.5em}{}
\titleformat{\subsection}{\normalfont\small\bfseries}{\thesubsection}{0.5em}{}
% Listes ultra-compactes
\setlist[itemize]{nosep, leftmargin=*, topsep=0pt, partopsep=0pt}
\setlist[enumerate]{nosep, leftmargin=*, topsep=0pt, partopsep=0pt}
% Tableaux compacts
\renewcommand{\arraystretch}{0.75}

\begin{document}
\scriptsize

\begin{multicols}{4}
% ===============================================
% PATTERN: Chapitre avec titre descriptif
% ===============================================
\section*{Chapitre N - NomDuChapitre}
\subsection*{1. PATTERN DEFINITION AVEC FORMULE}

{\bf Déf.}: Objet $\alpha$ de type $X$ satisfait propriété:
$$f(x) = expression$$

Texte explicatif $g(y) = autreExpr$ est le {\bf nom important}.

{\bf Méthode}: 1. Faire action1; 2. Calculer chose; 3. Résoudre equation

{\bf Ex.}: Déterminer propriété de $M = \begin{bmatrix} a & b \\ c & d \end{bmatrix}$

{\bf Sol.}:
$$M - \alpha I = \begin{bmatrix} a-\alpha & b \\ c & d-\alpha \end{bmatrix}$$

$$f(\alpha) = calcul1$$

$$= calcul2$$

$$= resultatFinal$$

{\bf Factorisation}: $expr = formeDeveloppee = 0$

{\bf Résultat}: $valeur_1 = res1, valeur_2 = res2$

{\bf Pièges}: Erreur courante 1; erreur 2; erreur 3. {\bf Note spéciale}: information.
\subsection*{2. PATTERN PROPRIETE SIMPLE}

{\bf Prop.}: Si $var_i$ propriété de $A$, alors propriété de $B$ sont $fonction(var_i)$. Autre propriété conservée.

{\bf Ex.}: $A$ a $var_1 = a, var_2 = b$. Question?

{\bf Sol.}: $res_1 = calcul1$, $res_2 = calcul2$. Description: $resultatA$. Autre info: $resultatB$.

{\bf Piège}: Conseil important!

{\bf Appl.}: Utilisation converge vers $val = nombre$ ce qui implique: $conclusion = calcul \approx valNum$
\subsection*{3. PATTERN ALGORITHME COMPLET}

{\bf But}: Objectif principal de la méthode avec $variable_1$ et propriété.

{\bf Algo}: Données $\vec{x}^0$ (condition), $param_0 = val$, $\epsilon$. Itérer: 1. $var^i = formule1$; 2. $var2^i = formule2$; 3. $res_i = formule3$; 4. Test $condition < \epsilon$ stop. Résultat: $res_i \to cible1$, $var^i \to cible2$.

{\bf Ex.}: N iter., $M = \begin{bmatrix} a & b \\ c & d \end{bmatrix}$, $\vec{v}^0 = vecteur^T$, $p_0 = val$.

{\bf Iter. 1}: a) $var^1 = calcul = res^T$; b) $mesure = valeur$, $norm^1 = vect^T$; c) $autre = calc$, $p_1 = nombre = val$

{\bf Iter. 2}: a) $var^2 = calcul = res^T$; b) $mesure = valeur$, $norm^2 = vect^T$; c) $p_2 \approx val$ converge vers $cible = val$.

{\bf Vitesse}: Ratio $r = |rapport|$; petit $r$ = rapide; erreur $\propto r^k$. Ex.: $|a| = v1$, $|b| = v2$ donc $r = v3$ (qualificatif).

{\bf Pièges}: Ne pas faire erreur1; oublier chose2; confondre concept3 et concept4.
\subsection*{4. PATTERN METHODE VARIANTE}

{\bf But}: Trouver la {\bf propriété spéciale} (en mode) de $X$.

{\bf Principe}: Appliquer technique à $transformation(X)$ (sans calculer explicitement $transformation$).

{\bf Vers quoi converge}: $resultat_1$ (description) = $relation$ (autre description)

{\bf Algo}

{\bf Différence avec autreMethode}: Au lieu de $step = simple$, on résout:
$$equation\_complexe = membre\_droit$$
(commentaire sur la différence)

{\bf Etapes}:

$var^i = formule1$; $res_i = formule2$; Résoudre $systeme$ (methode1, methode2, etc.); {\bf Test stop}: $condition < seuil$;

{\bf Ex.}

{\bf Éno.}: Effectuer N itérations de nomMethode pour $M = \begin{bmatrix} a & b \\ c & d \end{bmatrix}$ avec $\vec{v}^0 = \begin{bmatrix} x \\ y \end{bmatrix}$ et $p_0 = 0$.

{\bf Sol.}:

{\bf Iter. 1}

{\bf 1}: Action
$$var^0 = calcul = \begin{bmatrix} x \\ y \end{bmatrix}$$

{\bf 2}: Résoudre $equation$
$$\begin{bmatrix} a & b \\ c & d \end{bmatrix}\begin{bmatrix} x_1 \\ x_2 \end{bmatrix} = \begin{bmatrix} v_1 \\ v_2 \end{bmatrix}$$

Par méthode:
$$\vec{res}^1 = \begin{bmatrix} val1 \\ val2 \end{bmatrix}$$

{\bf 3}: Calculer $metrique_1$
$$metrique_1 = formuleCalcul = resultat$$

{\bf Iter. 2}

{\bf 1}: Action
$$mesure = calcul$$

{\bf 2}: Résoudre $equation2$

{\bf 3}: Calculer $metrique_2$

{\bf Convergence} (après M itérations): $var \approx valeurFinale$

{\bf Déduction}: Conclusion recherchée de $X$ est:
$$propriete = calcul \approx resultatNum$$

{\bf Vitesse de convergence}

Généralement {\bf COMPARATIF} que autreMethode!

{\bf Ratio mesure}: $r = formuleRatio$

{\bf Ex.}: Si $param_1 = v1$ et $param_2 = v2$, alors $r = v3$ donc convergence {\bf qualificatif} (N itérations)

{\bf Pièges à éviter}

$variableConvergee$ est la propriété de $A$; Vraie valeur de $B$: $formule = relation$;

\subsection*{5. PATTERN CONCEPT COURT COMPACT}
{\bf NomConcept}: $notation(X) = formule$. {\bf AutreConcept}: $condition$ ssi $critere < seuil$.
{\bf Ex.}: $M = \begin{bmatrix} a & b \\ c & d \end{bmatrix}$ question? Propriété donc $res_1 = v1, res_2 = v2$. $mesure(M) = val < seuil$ donc {\bf Conclusion}.
{\bf ATTENTION}: Point important dépend de $varA$, PAS $varB$. Calculer $formule$ et vérifier $condition$.

{\bf Pièges à éviter}

{\bf Confondre} conceptA de $X$ avec celui de $Y$ ou $Z$; {\bf Oublier} que $propriete$ peut être vraie même si condition2; {\bf Cas spécial}: Description cas = resultat;
\subsection*{6. PATTERN SYSTEME AVEC JACOBIENNE}

{\bf Système}: $\vec{x} = fonction(\vec{x})$

{\bf NomConcept}: $\vec{r}$ tel que $\vec{r} = fonction(\vec{r})$

{\bf Algo}: $\vec{x}^{k+1} = fonction(\vec{x}^k)$ (commentaire)

{\bf Critère propriété}: $\vec{r}$ est qualificatif si $mesure(J(\vec{r})) < 1$

où $J(\vec{r})$ est la {\bf matrice derivées} de $fonction$ évaluée en $\vec{r}$:

$$J(\vec{x}) = \begin{bmatrix}
\frac{\partial f_1}{\partial x_1} & \frac{\partial f_1}{\partial x_2} & \cdots \\
\frac{\partial f_2}{\partial x_1} & \frac{\partial f_2}{\partial x_2} & \cdots \\
\vdots & \vdots & \ddots
\end{bmatrix}$$

{\bf Ex.}

{\bf Éno.}: Soit le système:
$$\begin{cases}
x_1 = expr1(x_2) \\
x_2 = expr2(x_1)
\end{cases}$$

{\bf a)} Trouver un objetRecherche

{\bf b)} Vérifier propriété

{\bf c)} Effectuer N iter depuis pointInit

{\bf Sol.}:

{\bf Partie a) Trouver objetRecherche}

{\bf Astuce}: Chercher cas "facile" où $x_1 = x_2$

Si $x_1 = x_2$, alors:
$$x_1 = expression$$

En appliquant opération:
$$calcul1 \Rightarrow calcul2 \Rightarrow x_1 = val$$

{\bf Vérification}:
$x_1 = calcul = val$; $x_2 = calcul = val$;

{\bf ObjetTrouve}: $\vec{r} = \begin{bmatrix} v_1 \\ v_2 \end{bmatrix}$

{\bf Partie b) Vérifier propriété}

{\bf 1}: Calculer la matrice derivées

$$fonction(\vec{x}) = \begin{bmatrix} expr1 \\ expr2 \end{bmatrix}$$

{\bf Dérivées partielles}:

$$\frac{\partial f_1}{\partial x_1} = formule$$

$$\frac{\partial f_1}{\partial x_2} = formule$$

$$\frac{\partial f_2}{\partial x_1} = formule$$

$$\frac{\partial f_2}{\partial x_2} = formule$$

{\bf Jacobienne}: $J(\vec{x}) = \begin{bmatrix} a & b \\ c & d \end{bmatrix}$. En (p1,p2): $J = \begin{bmatrix} v1 & v2 \\ v3 & v4 \end{bmatrix}$. $equationCarac = 0 \Rightarrow \lambda = \pm expr$, $|\lambda| = val < 1$ donc {\bf CONCLUSION}.

{\bf Partie c) N itérations depuis ptInit}

{\bf k=0}: $x_1^0 = v0$, $x_2^0 = v0$; {\bf k=1}: $x_1^1 = v1$, $x_2^1 = v1$; {\bf k=2}: $x_1^2 = v2$, $x_2^2 = v2$; {\bf k=3}: $x_1^3 = v3$, $x_2^3 = v3$; {\bf k=4}: $x_1^4 = v4$, $x_2^4 = v4$; {\bf k=5}: $x_1^5 = v5$, $x_2^5 = v5$

{\bf Observation}: Description comportement convergeant vers (pt, pt) (explication)

{\bf ATTENTION ERREURS IMPORTANTES!}

{\bf ERREUR TYPE}: Description erreur; {\bf Oublier} calcul important; {\bf Erreur} dans étape (règle!); {\bf Confondre} conceptA ($cond < 1$) et conceptB ($cond > 1$);
\subsection*{7. PATTERN METHODE ITERATIVE SYSTEME}

{\bf Système}: $A\vec{x} = \vec{b}$

{\bf Principe}: Isoler $x_i$ dans equation $i$ en utilisant valeurs de iteration {\bf qualificatif}

{\bf Formule générale}:

$$x_i^{k+1} = formuleGenerale$$

{\bf Condition convergence}: Matrice $A$ à {\bf proprieteRequise}:

$$|terme_{ii}| > sommeAutres \quad \forall i$$

{\bf Ex.}

{\bf Éno.}: Résoudre par nomMethode (N iter) depuis $\vec{x}^0 = vectInit^T$:

$$\begin{aligned}
c_1 x_1 + c_2 x_2 + c_3 x_3 &= b_1 \\
c_4 x_1 + c_5 x_2 + c_6 x_3 &= b_2 \\
c_7 x_1 + c_8 x_2 + c_9 x_3 &= b_3
\end{aligned}$$

{\bf Sol.}:

{\bf Vérification préliminaire (nomTest)}

{\bf Ligne 1}: Diag $|c_1| = v1$, Somme $|c_2| + |c_3| = v2$, Test: $v1 > v2$

{\bf Ligne 2}: Diag $|c_5| = v1$, Somme $|c_4| + |c_6| = v2$, Test: $v1 > v2$

{\bf Ligne 3}: Diag $|c_9| = v1$, Somme $|c_7| + |c_8| = v2$, Test: $v1 > v2$

{\bf Conclusion test!}

{\bf Formules iteration}

$$\begin{cases}
x_1^{k+1} = formule1 \\
x_2^{k+1} = formule2 \\
x_3^{k+1} = formule3
\end{cases}$$

{\bf Iter. 0 (initial)}

$$\vec{x}^0 = \begin{bmatrix} v_1 \\ v_2 \\ v_3 \end{bmatrix}$$

{\bf Iter. 1}

$$x_1^1 = calcul = fraction \approx decimal$$

$$x_2^1 = calcul = fraction = decimal$$

$$x_3^1 = calcul = fraction \approx decimal$$

$$\vec{x}^1 = \begin{bmatrix} v_1 \\ v_2 \\ v_3 \end{bmatrix}$$

{\bf Iter. 2}

$$x_1^2 = calcul$$
$$= calcul2 = fraction \approx decimal$$

$$x_2^2 = calcul = fraction \approx decimal$$

$$x_3^2 = calcul$$
$$= calcul2 = fraction \approx decimal$$

$$\vec{x}^2 = \begin{bmatrix} v_1 \\ v_2 \\ v_3 \end{bmatrix}$$

{\bf Iter. 3}

$$x_1^3 \approx val, \quad x_2^3 \approx val, \quad x_3^3 \approx val$$

$$\vec{x}^3 = \begin{bmatrix} v_1 \\ v_2 \\ v_3 \end{bmatrix}$$

{\bf Convergence vers}: $\vec{x}^* = \begin{bmatrix} v_1 \\ v_2 \\ v_3 \end{bmatrix}$

{\bf Cas casSpecial}

{\bf Problème}: Si $terme_{ii} = 0$, on ne peut pas faire operation!

{\bf Sol.}: {\bf Action} (description)

{\bf Ex.}: Voir Exercice N (section suivante)

{\bf Pièges à éviter}

{\bf Methode1}: Description règle; {\bf Methode2}: Description différence;

{\bf CasProbleme}: Ne PAS faire action! Action correcte.

{\bf Erreur calcul}: Vérifier chaque étape (outil!)

{\bf Oublier} de vérifier condition préalable
\subsection*{8. PATTERN METHODE VARIANTE ITERATIVE}

{\bf Principe}: Comme autreMethode, mais on utilise valeurs {\bf modificateur} (iteration $k+1$)

{\bf Formule}:

$$x_i^{k+1} = formuleModifiee$$

{\bf Avantage}: Généralement {\bf COMPARATIF} que autreMethode (facteur× plus rapide typiquement)

{\bf Condition convergence}: Identique à autreMethode (critere)

{\bf Ex. (même système que autreMethode)}

{\bf Formules iteration} (noter différence):

$$\begin{cases}
x_1^{k+1} = formule\_avec\_k \\
x_2^{k+1} = formule\_mixte\_k\_kplus1 \\
x_3^{k+1} = formule\_avec\_kplus1
\end{cases}$$

{\bf Iter.}: 1: $\vec{x}^1 = vect^T$; 2: $\vec{x}^2 = vect^T$; 3: $\vec{x}^3 \approx vect^T$. {\bf Conclusion en N iter!} (vs M+ autreMethode)

{\bf Comparaison Methode1 vs Methode2}

{\bf Methode1 Iter. N}: $(v1, v2, v3)$, Mesure $\sim p\%$

{\bf Methode2 Iter. N}: $(v1, v2, v3)$, Mesure {\bf 0\%} (qualificatif!)

{\bf Pièges à éviter}

{\bf Erreur commune} consequence!; {\bf Conseil}: Description action; {\bf Erreur} description problème;
\subsection*{9. PATTERN COMPARAISON INLINE}

{\bf Critere Methode1}: Convergence si $condition1$; Propriété suffisante: descriptionProp; Vitesse: Qualificatif; Utilise $k+1$: Non; Coût iteration: Identique; Parallélisable: Oui

{\bf Critere Methode2}: Convergence si $condition2$; Propriété suffisante: descriptionProp; Vitesse: {\bf Qualificatif} ($\sim$facteur× Methode1); Utilise $k+1$: Oui; Coût iteration: Identique; Parallélisable: Non (sequentiel)

{\bf Ex.: NomExempleComplexe}

{\bf Système initial} (PROBLEME):

$$\begin{aligned}
E_1: \quad equation1 &= val \\
E_2: \quad equation2 &= val \\
E_3: \quad equation3 &= val \\
E_4: \quad equation4 &= val
\end{aligned}$$

{\bf Problème}: Ligne N a $terme = 0$ (description!)

{\bf Vérification test}:

{\bf L1}: Diag v1, Somme v2, Crit: Non; {\bf L2}: Diag v1, Somme v2, Crit: Non; {\bf L3}: Diag v1, Somme v2, Crit: Non; {\bf L4}: Diag v1, Somme v2, Crit: Non

{\bf Conclusion lignes} pas propriété donc {\bf consequence}

{\bf Sol.: NomSolution}

{\bf Stratégie}: Placer le {\bf element important} de chaque equation sur position cible

$E_1$: Max = $|val|$ (pour $var_N$) donc Ligne N; $E_2$: Max = $|val|$ (pour $var_M$) donc Ligne M; $E_3$: Max = $|val|$ (pour $var_K$) donc Ligne K; $E_4$: Max = $|val|$ (pour $var_J$) donc Ligne J;

{\bf Système reordonne} (RESOLU):

$$\begin{aligned}
nouvelleEq1 &= val \\
nouvelleEq2 &= val \\
nouvelleEq3 &= val \\
nouvelleEq4 &= val
\end{aligned}$$

{\bf Vérification test}:

{\bf L1}: Diag v1, Somme v2, Crit: Oui; {\bf L2}: Diag v1, Somme v2, Crit: Oui; {\bf L3}: Diag v1, Somme v2, Crit: Oui; {\bf L4}: Diag v1, Somme v2, Crit: Oui

{\bf Toutes lignes} satisfont propriété donc {\bf conclusion positive!}
\subsection*{10. PATTERN PIEGES ERREURS LISTE}

{\bf TOP N des erreurs à éviter}

{\bf ErreurType1}: Description erreur; {\bf ErreurType2}: Confondre conceptA et conceptB (utiliser $notation1$ vs $notation2$); {\bf ErreurType3}: Confondre $mesureA$ avec $mesureB$; {\bf ErreurType4}: Oublier $relation = formule$; {\bf ErreurType5}: Oublier de faire action dans methode; {\bf ErreurType6}: Faire mauvaise action au lieu de bonne; {\bf ErreurType7}: Ne pas vérifier condition avant action; {\bf ErreurType8}: Oublier operation pour cas special; {\bf ErreurType9}: Conclure trop vite (faire au moins N iterations); {\bf ErreurType10}: Choisir mauvais element dans methode;

{\bf Checklist}

{\bf Toujours} action importante pour methode; {\bf Ne jamais} confondre methodeA et methodeB (detail!); {\bf Faire} action recommandée (plus clair); {\bf Vérifier} calculs avec outil; {\bf Attention} aux probleme fréquent (source erreur numero 1); {\bf Ne pas} perdre temps sur action inutile (conseil); {\bf Utiliser} ressources fournies avec examen;

% ===============================================
% PATTERN: Nouveau chapitre différent
% ===============================================
\section*{Chapitre M - AutreTheme}
\subsection*{1. PATTERN THEOREME FONDAMENTAL}

{\bf NomTheoreme fondamental}

{\bf Théorème}: Par $n+1$ elements type $(x_0, y_0), \ldots, (x_n, y_n)$, il passe {\bf un unique} objet de degré $\leq n$.

{\bf Implications pratiques}

Methode1, Methode2, et Methode3 donnent le {\bf même résultat}; Seule la {\bf forme} diffère (type de formule); On peut vérifier propriété en appliquant transformation;

{\bf Ex. examen}

{\bf Question}: Les objets de Methode1 et Methode2 passant par N points sont-ils identiques?

{\bf Réponse structurée}:
REPONSE, ils sont identiques (même objet); Raison: theoreme propriété (N points donc unique objet degré $\leq$ M); Seule la formulation change (Methode1 vs Methode2); En appliquant transformation, on obtient même expression;

{\bf Piège à éviter}

Ne PAS confondre "même objet" et "même formule"; Ne PAS dire qu'ils donnent résultats différents;

\end{multicols}

\end{document}
