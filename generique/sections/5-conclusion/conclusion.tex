% ============================================================
% CONCLUSION
% ============================================================
% Instructions pour rédiger cette section :
% - Résumez les points principaux du rapport
% - Rappelez les objectifs et ce qui a été accompli
% - Mettez en évidence les résultats les plus importants
% - Proposez des perspectives ou améliorations futures
% - Gardez cette section concise (1-2 pages maximum)
% ============================================================

\section{Conclusion}
\label{sec:conclusion}

% TODO: Remplacez ce texte par votre conclusion

\subsection{Synthèse}

% TODO: Résumez les points principaux du rapport

Ce rapport a présenté [résumer brièvement le contenu principal].

Les principaux points abordés sont :
\begin{itemize}
	\item Point principal 1
	\item Point principal 2
	\item Point principal 3
\end{itemize}

\subsection{Accomplissements}

% TODO: Indiquez ce qui a été accompli par rapport aux objectifs

Les objectifs fixés en introduction ont été atteints :
\begin{itemize}
	\item Objectif 1 : [statut/résultat]
	\item Objectif 2 : [statut/résultat]
	\item Objectif 3 : [statut/résultat]
\end{itemize}

\subsection{Résultats clés}

% TODO: Mettez en évidence les résultats les plus importants

Les résultats les plus significatifs de ce travail sont :
\begin{itemize}
	\item Résultat important 1
	\item Résultat important 2
	\item Résultat important 3
\end{itemize}

\subsection{Perspectives}

% TODO: Proposez des améliorations ou travaux futurs

Ce travail pourrait être prolongé ou amélioré de plusieurs façons :
\begin{itemize}
	\item Amélioration possible 1
	\item Amélioration possible 2
	\item Piste de recherche future
\end{itemize}

% TODO: Phrase de clôture finale

En conclusion, ce rapport démontre que...
