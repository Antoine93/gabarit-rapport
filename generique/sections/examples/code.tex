% ============================================================
% EXEMPLES DE CODE SOURCE MATLAB
% ============================================================

\subsection*{Code source MATLAB}

Le style est préconfiguré pour afficher du code MATLAB avec coloration syntaxique et support des accents.

\subsubsection*{Code inline}

Pour insérer du code dans une phrase, utilisez \texttt{\textbackslash lstinline} :

\begin{verbatim}
La fonction \lstinline{disp('Bonjour')} affiche du texte.
\end{verbatim}

Résultat : La fonction \lstinline{disp('Bonjour')} affiche du texte.

\subsubsection*{Bloc de code simple}

Pour un bloc de code avec numérotation automatique :

\begin{lstlisting}[caption={Calcul de la moyenne}, label={code:moyenne}]
% Calcul de la moyenne d'un vecteur
function moyenne = calculer_moyenne(vecteur)
	% Entrée : vecteur - vecteur de nombres
	% Sortie : moyenne - moyenne arithmétique

	n = length(vecteur);
	somme = sum(vecteur);
	moyenne = somme / n;

	disp(['La moyenne est : ', num2str(moyenne)]);
end
\end{lstlisting}

\subsubsection*{Code depuis un fichier externe}

Pour inclure un fichier MATLAB complet :

\begin{verbatim}
\lstinputlisting[caption={Script principal},
                 label={code:principal}]{code/main.m}
\end{verbatim}

\subsubsection*{Code avec options personnalisées}

Vous pouvez personnaliser l'affichage pour un bloc spécifique :

\begin{lstlisting}[
	caption={Fonction avec accents français},
	label={code:accents},
	numbers=left,
	frame=single,
	backgroundcolor=\color{matlabbg}
]
% Résolution d'équation
function solution = resoudre_equation(a, b, c)
	% Résout l'équation ax² + bx + c = 0

	delta = b^2 - 4*a*c;

	if delta < 0
		disp('Pas de solution réelle');
		solution = [];
	elseif delta == 0
		solution = -b / (2*a);
		disp(['Solution unique : ', num2str(solution)]);
	else
		x1 = (-b + sqrt(delta)) / (2*a);
		x2 = (-b - sqrt(delta)) / (2*a);
		solution = [x1, x2];
		disp(['Deux solutions : ', num2str(x1), ' et ', num2str(x2)]);
	end
end
\end{lstlisting}

\subsubsection*{Code sans numérotation}

Pour désactiver les numéros de ligne :

\begin{lstlisting}[caption={Code court}, numbers=none]
% Vecteur de 1 à 10
v = 1:10;
plot(v, v.^2);
title('Fonction carré');
xlabel('x');
ylabel('y');
\end{lstlisting}

\subsubsection*{Extraire une portion de fichier}

Pour n'inclure que certaines lignes d'un fichier :

\begin{verbatim}
\lstinputlisting[firstline=10, lastline=25,
                 caption={Extrait du fichier}]{code/main.m}
\end{verbatim}

\subsubsection*{Référencer du code}

Pour référencer un listing dans le texte :

\begin{verbatim}
Le Code~\ref{code:moyenne} montre comment calculer une moyenne.
Voir le Code~\ref{code:accents} pour un exemple avec accents.
\end{verbatim}

Résultat : Le Code~\ref{code:moyenne} montre comment calculer une moyenne.

\subsubsection*{Options disponibles pour lstlisting}

\begin{itemize}
	\item \texttt{caption=\{...\}} : Légende du code
	\item \texttt{label=\{...\}} : Label pour référence
	\item \texttt{numbers=left/right/none} : Position des numéros de ligne
	\item \texttt{firstline=N} : Première ligne à inclure
	\item \texttt{lastline=N} : Dernière ligne à inclure
	\item \texttt{frame=single/none} : Cadre autour du code
	\item \texttt{breaklines=true/false} : Coupure automatique des lignes longues
	\item \texttt{basicstyle=\textbackslash tiny} : Taille du texte (tiny, small, normalsize)
\end{itemize}

\subsubsection*{Exemple complet avec algorithme}

\begin{lstlisting}[caption={Méthode de Newton-Raphson}, label={code:newton}]
function racine = newton_raphson(f, df, x0, tol, max_iter)
	% Méthode de Newton-Raphson pour trouver les racines
	% Entrées:
	%   f        - fonction à résoudre
	%   df       - dérivée de f
	%   x0       - estimation initiale
	%   tol      - tolérance
	%   max_iter - nombre maximal d'itérations
	% Sortie:
	%   racine   - racine approximative

	x = x0;

	for i = 1:max_iter
		fx = f(x);
		dfx = df(x);

		% Éviter la division par zéro
		if abs(dfx) < eps
			error('Dérivée nulle - échec de la méthode');
		end

		% Calcul de la nouvelle approximation
		x_new = x - fx/dfx;

		% Vérification de la convergence
		if abs(x_new - x) < tol
			racine = x_new;
			fprintf('Convergence en %d itérations\n', i);
			return;
		end

		x = x_new;
	end

	warning('Nombre maximal d''itérations atteint');
	racine = x;
end
\end{lstlisting}

\subsubsection*{Formules mathématiques dans les commentaires}

Pour inclure des formules LaTeX dans les commentaires MATLAB, utilisez \texttt{escapechar} :

\begin{lstlisting}[caption={Code avec formules mathématiques}, escapechar=|]
% Résolution d'équation
function solution = resoudre_equation(a, b, c)
	% Résout l'équation |$ax^2 + bx + c = 0$|

	% Calcul du discriminant |$\Delta = b^2 - 4ac$|
	delta = b^2 - 4*a*c;

	if delta < 0
		% Aucune solution réelle (|$\Delta < 0$|)
		solution = [];
	elseif delta == 0
		% Solution unique : |$x = -\frac{b}{2a}$|
		solution = -b / (2*a);
	else
		% Deux solutions : |$x = \frac{-b \pm \sqrt{\Delta}}{2a}$|
		x1 = (-b + sqrt(delta)) / (2*a);
		x2 = (-b - sqrt(delta)) / (2*a);
		solution = [x1, x2];
	end
end
\end{lstlisting}

\textbf{Note :} Avec \texttt{escapechar=|}, tout ce qui est entre \texttt{|...|} sera interprété comme du LaTeX.

\subsubsection*{Support des accents}

Le style supporte automatiquement les accents français dans les commentaires :

\begin{lstlisting}[caption={Test des accents}]
% Caractères accentués: é è ê ë à â î ï ô ç ù û
% Création d'un vecteur
vecteur = [1, 2, 3];

% Affichage avec accents
disp('Résultat de l''opération:');
disp('Température: 25°C');
\end{lstlisting}
