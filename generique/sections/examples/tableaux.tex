% ============================================================
% EXEMPLES DE TABLEAUX
% ============================================================

\subsection*{Tableaux}

\subsubsection*{Tableau simple}

Un tableau simple avec bordures et légende :

\begin{table}[H]
	\centering
	\caption{Résultats expérimentaux}
	\label{tab:exemple1}
	\begin{tabular}{|c|c|c|}
		\hline
		\textbf{Essai} & \textbf{Température (°C)} & \textbf{Pression (kPa)} \\
		\hline
		1 & 25.3 & 101.2 \\
		2 & 30.1 & 102.5 \\
		3 & 28.7 & 100.8 \\
		\hline
	\end{tabular}
\end{table}

\textbf{Code :}
\begin{verbatim}
\begin{table}[H]
  \centering
  \caption{Résultats expérimentaux}
  \label{tab:exemple1}
  \begin{tabular}{|c|c|c|}
    \hline
    \textbf{Essai} & \textbf{Temp.} & \textbf{Pression} \\
    \hline
    1 & 25.3 & 101.2 \\
    \hline
  \end{tabular}
\end{table}
\end{verbatim}

\subsubsection*{Tableau sans bordures (style professionnel)}

Les tableaux professionnels utilisent souvent moins de lignes :

\begin{table}[H]
	\centering
	\caption{Comparaison des méthodes numériques}
	\label{tab:exemple2}
	\begin{tabular}{lcc}
		\hline
		\textbf{Méthode} & \textbf{Précision} & \textbf{Temps (s)} \\
		\hline
		Euler explicite & $10^{-3}$ & 0.12 \\
		Runge-Kutta 4 & $10^{-6}$ & 0.45 \\
		Adams-Bashforth & $10^{-5}$ & 0.28 \\
		\hline
	\end{tabular}
\end{table}

\subsubsection*{Tableau avec alignement mixte}

Colonnes alignées différemment (l = gauche, c = centre, r = droite) :

\begin{table}[H]
	\centering
	\caption{Propriétés des matériaux}
	\label{tab:exemple3}
	\begin{tabular}{lrr}
		\hline
		\textbf{Matériau} & \textbf{Densité (kg/m³)} & \textbf{Module de Young (GPa)} \\
		\hline
		Acier & 7850 & 200 \\
		Aluminium & 2700 & 69 \\
		Cuivre & 8960 & 130 \\
		Titane & 4506 & 116 \\
		\hline
	\end{tabular}
\end{table}

\subsubsection*{Tableau avec cellules fusionnées}

Pour fusionner des cellules, utilisez \texttt{\textbackslash multicolumn} :

\begin{table}[H]
	\centering
	\caption{Résultats par catégorie}
	\label{tab:exemple4}
	\begin{tabular}{|l|c|c|c|}
		\hline
		\multicolumn{1}{|c|}{\textbf{Catégorie}} & \multicolumn{3}{c|}{\textbf{Résultats}} \\
		\hline
		& \textbf{Min} & \textbf{Moy} & \textbf{Max} \\
		\hline
		Série A & 12.3 & 15.7 & 18.9 \\
		Série B & 10.1 & 14.2 & 17.5 \\
		Série C & 11.8 & 16.3 & 20.1 \\
		\hline
	\end{tabular}
\end{table}

\subsubsection*{Référencer un tableau}

Pour référencer un tableau dans le texte, utilisez \texttt{\textbackslash ref\{label\}} :

Le Tableau~\ref{tab:exemple1} présente les résultats expérimentaux.
Comme on peut le voir dans le Tableau~\ref{tab:exemple2}, la méthode Runge-Kutta 4 offre la meilleure précision.

\subsubsection*{Paramètres de positionnement}

\begin{itemize}
	\item \texttt{[H]} : Exactement ici (nécessite le package \texttt{float})
	\item \texttt{[h]} : Approximativement ici
	\item \texttt{[t]} : En haut de la page
	\item \texttt{[b]} : En bas de la page
	\item \texttt{[p]} : Sur une page dédiée aux flottants
\end{itemize}
