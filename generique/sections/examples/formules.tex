% ============================================================
% EXEMPLES DE FORMULES MATHÉMATIQUES
% ============================================================

\subsection*{Formules mathématiques}

\subsubsection*{Formules en ligne (inline)}

Les formules en ligne s'insèrent directement dans le texte avec \texttt{\$...\$}.

Exemples :
\begin{itemize}
	\item L'équation $E = mc^2$ est célèbre.
	\item La solution est $x = \frac{-b \pm \sqrt{b^2 - 4ac}}{2a}$.
	\item Pour tout $n \in \mathbb{N}$, on a $\sum_{i=1}^{n} i = \frac{n(n+1)}{2}$.
\end{itemize}

\subsubsection*{Formules centrées (display mode)}

Pour des formules importantes, utilisez l'environnement \texttt{equation} ou \texttt{\$\$...\$\$}.

\textbf{Avec numérotation :}
\begin{equation}
	\int_{a}^{b} f(x) \, dx = F(b) - F(a)
\end{equation}

\begin{equation}
	\nabla \times \mathbf{E} = -\frac{\partial \mathbf{B}}{\partial t}
\end{equation}

\textbf{Sans numérotation :}
\begin{equation*}
	e^{i\pi} + 1 = 0
\end{equation*}

\subsubsection*{Systèmes d'équations}

Utilisez l'environnement \texttt{align} pour aligner plusieurs équations :

\begin{align}
	x + y &= 5 \\
	2x - y &= 1
\end{align}

Ou sans numérotation avec \texttt{align*} :

\begin{align*}
	\frac{\partial u}{\partial t} &= \alpha \frac{\partial^2 u}{\partial x^2} \\
	u(0,t) &= u(L,t) = 0 \\
	u(x,0) &= f(x)
\end{align*}

\subsubsection*{Matrices}

Les matrices utilisent l'environnement \texttt{pmatrix}, \texttt{bmatrix}, ou \texttt{vmatrix} :

\begin{equation}
	A = \begin{bmatrix}
		a_{11} & a_{12} & a_{13} \\
		a_{21} & a_{22} & a_{23} \\
		a_{31} & a_{32} & a_{33}
	\end{bmatrix}
\end{equation}

\begin{equation}
	\det(A) = \begin{vmatrix}
		1 & 2 & 3 \\
		4 & 5 & 6 \\
		7 & 8 & 9
	\end{vmatrix}
\end{equation}

\subsubsection*{Symboles mathématiques courants}

\begin{itemize}
	\item Fractions : $\frac{a}{b}$, $\dfrac{a}{b}$
	\item Racines : $\sqrt{x}$, $\sqrt[n]{x}$
	\item Exposants et indices : $x^2$, $x_i$, $x_i^2$
	\item Sommes et produits : $\sum_{i=1}^{n}$, $\prod_{i=1}^{n}$
	\item Intégrales : $\int$, $\iint$, $\iiint$, $\oint$
	\item Limites : $\lim_{x \to \infty}$
	\item Dérivées : $\frac{df}{dx}$, $\frac{\partial f}{\partial x}$
	\item Vecteurs : $\vec{v}$, $\mathbf{v}$
	\item Ensembles : $\mathbb{N}$, $\mathbb{Z}$, $\mathbb{Q}$, $\mathbb{R}$, $\mathbb{C}$
	\item Opérateurs logiques : $\forall$, $\exists$, $\in$, $\notin$, $\subseteq$
	\item Flèches : $\rightarrow$, $\Rightarrow$, $\leftrightarrow$, $\Leftrightarrow$
\end{itemize}
